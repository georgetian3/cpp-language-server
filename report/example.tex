\section{辅助说明}

a: 图(a)是代码高亮和补全程序的一个基本页面展示。当用户输入代码时,输入的代码会根据相应类型进行颜色的变换(高亮),如class是亮绿色,#include是粉紫色等等。

b: 对于类的成员变量和成员函数,只会在用户在对应的类对象后输入"."或者"->",相应的成员函数和变量会在suggestions中提示。用户可以通过ctrl+显示按键的方式,对代码进行补全,如图(b)所示。

c: 图(c)展示了程序中实现了对定义域(或者说可调用域)和可操作域的实现。当函数或者是变量被定义或者声明后,用户再次在相同定义域domain中试图输入时,会有提示和补全信息。

d: 而对于有着特定定义域的变量来说,它们不会在除自身可用定义域以外的其他定义域中被提示可补全,如图(d)中的in_function。