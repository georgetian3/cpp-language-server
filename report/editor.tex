\section{代码高亮与提示}

\subsection{前端}

代码高亮与提示的可视化是通过网页在 \fileref{editor/} 下的 HTML, JS 与 CSS 文件实现的。其中使用了 Bootstrap 做格式化,与定义了一个 HTML 元素、提供了文本输入的基本框架与格式的 \href{https://github.com/WebCoder49/code-input}{\code{code-input}} 。此页面中包括文本编译器的几个基本功能,其描述以及实现方法如下:

\begin{enumerate}
    \item 代码高亮:\code{code-input} 提供了一个 JS 接口,让开发者定义一个函数,其参数为包含用户代码输入的文本框。用户每次改变文本框的内容会调用这个函数,并由这个函数根据用户的输入而提供包含高亮的 HTML 代码。此函数在 \fileref{editor/editor.js} 中的 \code{handle_update} 函数实现。它将用户的输入发给后端,并接收后端的响应而更新网页上的高亮状态。后端返回的 HTML 中包括每个记号本身与其类名。\code{editor/editor.css} 根据类名进行高亮。高亮的颜色与 VSCode 的 Microsoft C/C++ 插件的颜色一致,即关键字为深蓝色、标识符为浅蓝色、数值型常量为浅绿色、字符(串)常量为橘色、运算符为白色、类名为绿色、引入语句为紫色、函数名及其调用为黄色、注释为深绿色。
    \item 代码提示:前端发给后端的信息中包括用户的光标目前所在的位置。后端根据光标的位置与已有的代码提供若干个提示,由 \code{handle_update} 显示这些提示。当用户通过点击或者使用快捷键选择某个提示后,\fileref{editor/editor.js} 中的 \code{autocomplete} 函数将进行补全。此外当用户输入 \lstinline`' " ( [ {` 中的一个字符后,它对应的字符将自动补全。
    \item 快捷键:用户可以使用制表键插入四个空格符号。使用 \code{ctrl + n} 使用序号为 \code{n} 的提示进行代码补全。
\end{enumerate}

\subsection{后端}

由于输入到语法分析器中的记号中不能包含注释,但是代码高亮中需要展示注释,所以传给语法分析器的词法分析器是继承 \code{lex.Lexer} 的 \code{NoCommentsLexer}。除了在调用 \code{token} 函数时它会跳过所有注释,\code{NoCommentsLexer} 与 \code{lex.Lexer} 的功能完全一样。

对于后端,需要利用词法分析得到的token,以及语法分析得到的ast。首先对于语法分析得到的ast,由于用户在输入代码的过程中,大多数时间不会形成一个完整的语法分析树,因此程序会保留上一个完整形成的语法分析树,直到下一个完整的分析树出现。因此,\underline{请确认输入是合法的c++源代码}。

后端主要实现的功能如下:
\begin{enumerate}
    \item 信息存储:对于一个ast树,递归地去寻找里面的函数,变量,类等信息,对于变量,会记录其的作用域为某个函数。对于类,会记录其的成员变量以及成员函数。对于变量,会记录其的数据类型。
    \item 代码补全:前端传给后端一个鼠标当前位置cursor,后端从cursor位置往前搜索直到第一个函数名,将当前的作用域定为这个函数。同时根据cursor获取源代码中待补全的片段,并在变量表中判断其是否为某个变量的前缀,如果是,补全之。当这个变量的类型是类时,且待补全片段中含有\code{.}或\code{->}时,后端会返回类表中与\code{.}或\code{->}片段前缀相符的成员变量或成员函数。
    \item 高亮提示:对于高亮提示,后端会将词法分析得到的token进行遍历,根据语法分析的结果分析其类型,并将类型返回前端。
\end{enumerate}


