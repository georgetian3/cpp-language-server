\section{词法分析}

\subsection{实现}

所有有关词法分析的代码存放在 \fileref{lexer} 下。标准中的第五章中定义了 C++ 的词法规则。其中描述的记号包括标识符(identifier)、关键字(keyword)、常量(literal)、运算符(operator),其中常量分为整数(integer)、字符(character)、浮点数(floating-point)、字符串(string)、布尔(boolean)、指针(pointer)、用户定义(user-defined)。各种记号的定义都存放在 \fileref{lexer} 下对应的 Python 文件中,并使用 PLY 的 \code{lex} 包进行结合与词法分析。开发过程中发现 \code{lex} 最后生成的主正则表达式中每种记号的顺序与这些在代码中规则定义的顺序可能不一样,导致比如 \code{char c = u8't'} 中的 \code{u8} 被识别为标识符,而正确的词法分析结果是整个 \code{u8't'} 是一个字符常量。因此改变了 \fileref{myply/lex.py} 中的 \code{lex} 函数,让用户明确地定义记号类型的识别顺序。因此此项目 \underline{必须使用 \fileref{myply} 下的 PLY 包}。

\subsection{难点}

阅读标准之前,我们以为 C++ 标识符可以简简单单地使用 \code{[_a-zA-Z][_a-zA-Z0-9]*} 的正则表达式匹配。但是 C++20 将可以作为标识符的符号大幅增加,甚至中文字与表情是合法的标识符字符。为了强调标准所允许的范围,以下显示的是符合标准的正则表达式:

\begin{lstlisting}[basicstyle=\ttfamily\tiny]
((?![\u0300-\u036F\u1DC0-\u1DFF\u20D0-\u20FF\uFE20-\uFE2F])([_a-zA-Z]|[\u00A8\u00AA\u00AD\u00AF\u00B2-\u00B5\u00B7-\u00BA\u00BC-\u00BE\u00C0-\u00D6\u00D8-\u00F6\u00F8-\u00FF\u0100-\u167F\u1681-\u180D\u180F-\u1FFF\u200B-\u200D\u202A-\u202E\u203F-\u2040\u2054\u2060-\u206F\u2070-\u218F\u2460-\u24FF\u2776-\u2793\u2C00-\u2DFF\u2E80-\u2FFF\u3004-\u3007\u3021-\u302F\u3031-\uD7FF\uF900-\uFD3D\uFD40-\uFDCF\uFDF0-\uFE44\uFE47-\uFFFD\U00010000-\U0001FFFD\U00020000-\U0002FFFD\U00030000-\U0003FFFD\U00040000-\U0004FFFD\U00050000-\U0005FFFD\U00060000-\U0006FFFD\U00070000-\U0007FFFD\U00080000-\U0008FFFD\U00090000-\U0009FFFD\U000A0000-\U000AFFFD\U000B0000-\U000BFFFD\U000C0000-\U000CFFFD\U000D0000-\U000DFFFD\U000E0000-\U000EFFFD])(([_a-zA-Z]|[\u00A8\u00AA\u00AD\u00AF\u00B2-\u00B5\u00B7-\u00BA\u00BC-\u00BE\u00C0-\u00D6\u00D8-\u00F6\u00F8-\u00FF\u0100-\u167F\u1681-\u180D\u180F-\u1FFF\u200B-\u200D\u202A-\u202E\u203F-\u2040\u2054\u2060-\u206F\u2070-\u218F\u2460-\u24FF\u2776-\u2793\u2C00-\u2DFF\u2E80-\u2FFF\u3004-\u3007\u3021-\u302F\u3031-\uD7FF\uF900-\uFD3D\uFD40-\uFDCF\uFDF0-\uFE44\uFE47-\uFFFD\U00010000-\U0001FFFD\U00020000-\U0002FFFD\U00030000-\U0003FFFD\U00040000-\U0004FFFD\U00050000-\U0005FFFD\U00060000-\U0006FFFD\U00070000-\U0007FFFD\U00080000-\U0008FFFD\U00090000-\U0009FFFD\U000A0000-\U000AFFFD\U000B0000-\U000BFFFD\U000C0000-\U000CFFFD\U000D0000-\U000DFFFD\U000E0000-\U000EFFFD])|[0-9])*)
\end{lstlisting}

\newpage

\subsection{创新点}

除了标识符以外,标准定义的其他若干种记号都有类似于标识符的复杂性,因此大大增加了我们的工作量。但是可以说,除了预处理与用户定义记号以外,我们的词法分析部分相当符合 C++20 标准。此外小组实现了一个可视化的网页界面,当把鼠标光标放到记号上的时候,会自动显示它的了行。整体的页面非常的间接且清晰明了。