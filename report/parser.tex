\section{文法分析}

\subsection{实现}

所有有关语法的代码存放在 \fileref{parser} 下。标准中的第六章到第十五章中都包含了有关语法的定义。我们最初试图实现所有语法规则,并且确实写除了绝大部分的语法规则,但由于 C++20 的语法规则巨多,并且调试比较繁琐,因此放弃了支持所有 C++20 规则的目标。我们在已有的基础上减掉了许多超过本次项目要求的功能,比如预处理、lambda 函数、模板等等。最后保留的功能包括 preprocessing、expressions、statements、declarations、classes,其上下文无关文法存放在对应的文件中。

\subsection{难点}

不管是词法分析还是语法分析的大作业部分,本小组一开始都是严格按照C++标准去进行实现的。在一步一步按照 C++20 标准实现语法分析部分的时候,最长的 declarations 文法足足写了有 800 多行代码。而在我们实现了那十个部分的文法后,发现工作量大带来的问题便是在调试的过程中会遇到不计其数的问题。首先是修改起来最简单的 typo 问题,有显性的,比如打错了字,也有隐性的,比如某一层文法或者产生式中多写了一个分号。在之后,便是各种文法冲突的问题。在试图解决文法冲突,并苦苦挣扎了将近一整个周末后,我们不得已选择抛弃掉部分 C++20 标准,选择按照大作业文档中的要求,保留基础文法,最终实现了语法高亮程序需要用的语法分析部分。